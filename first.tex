%kelompok 4 D4 TI-2D
%Ayu Permata Sari        1154022
%Librantara Erlangga     1154071
%Martin Luter Zega       1154120
%Putri Aulia Ramadhanie  1154096
%Ryan Hafizh Herdiana    1154067
%Copyright (c) 2017 Copyright Holder All Rights Reserved.

\section{Pyshp}
------

\subsection{basic usage}
berikut ini adalah basic penggunaan module pyshp di python

Membuat file pada pyshp
  \begin{verbatim}
  import shapefile
  a = shapefile.Writer(shapeType=1)
  a.field('nama','C','40')
  a.field('alamat','C','255')
  a.save('namafile.shp')
  \end{verbatim}

Mengedit atau Menambah record
\begin{verbatim}
  import shapefile
  a = shapefile.Editor(shapefile='namafile.shp')
  a.record('politeknis pos','jl.sarijadi')
  a.point(-6.8731953,107,5737873,0,0)
  a.save('namafile')
\end{verbatim}

Menghapus record
\begin{verbatim}
  import shapefile
  a = shapefile.Editor('namafile.shp')
  a.shape(0) //masukan data ke berapa //karena array jadi dimulai dari 0
  a.delete(0)
  a.save('namafile')
\end{verbatim}

Membaca record
\begin{verbatim}
  import shapefile
  a = shapefile.Reader('namafile.shp')
  a.records() //menampilkan semua record
  a.record(0) //spesifik record
\end{verbatim}
