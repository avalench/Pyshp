%kelompok 4 D4 TI-2D
%Ayu Permata Sari        1154022
%Librantara Erlangga     1154071
%Martin Luter Zega       1154120
%Putri Aulia Ramadhanie  1154096
%Ryan Hafizh Herdiana    1154067
%Copyright (c) 2017 Copyright Holder All Rights Reserved.

\section{Pyshp}
Shapefile adalah file yang berisi domain map dataset dan dapat dibuka dengan aplikasi-aplikasi tertentu yang memiliki fitur GIS didalamnya.
Python adalah bahasa pemrograman yang dapat digunakan untuk membuka shapefile tersebut.
Pyshp adalah library python yang berfungsi agar bisa membaca shapefile, salah satunya format (.shp).
Pip adalah Package Management System yang berfungsi untuk meng-install dan me-manage paket software yang ada didalam Python.
Metode yang saya lakukan diatas adalah menghitung jumlah shape yang tersedia didalam file shapefile tersebut.


\subsection{basic usage}
berikut ini adalah basic penggunaan module pyshp di python

Membuat file pada pyshp
  \begin{verbatim}
  import shapefile
  a = shapefile.Writer(shapeType=1)
  a.field('nama','C','40')
  a.field('alamat','C','255')
  a.save('namafile.shp')
  \end{verbatim}

Mengedit atau Menambah record
\begin{verbatim}
  import shapefile
  a = shapefile.Editor(shapefile='namafile.shp')
  a.record('politeknis pos','jl.sarijadi')
  a.point(-6.8731953,107,5737873,0,0)
  a.save('namafile')
\end{verbatim}

Menghapus record
\begin{verbatim}
  import shapefile
  a = shapefile.Editor('namafile.shp')
  a.shape(0) //masukan data ke berapa //karena array jadi dimulai dari 0
  a.delete(0)
  a.save('namafile')
\end{verbatim}

Membaca record
\begin{verbatim}
  import shapefile
  a = shapefile.Reader('namafile.shp')
  a.records() //menampilkan semua record
  a.record(0) //spesifik record
\end{verbatim}

\subsection
Cara Menghitung jumlah record dari File .shp menggunakan python
Install Python terlebih dahulu
Install PIP di python
Buka cmd lalu ketik: Python –m pip install pyshp ,lalu enter
Kemudian upgrade pip dengan mengetikkan di cmd: Python –m pip install - -upgrade pip ,lalu enter
Lalu masuk ke python
Ketikkan script berikut: import shapefile ,klik enter f = shapefile.Reader("lokasi file .shp") ,klik enter shapes = f.shapes() ,klik enter print len (shapes)
Lalu klik enter,maka akan keluar hasil nya,disini kita menghitung jumlah record yang ada di dalam file .shp Membuka file shapefile juga dapat kita lakukan dengan membuat file .py(python),dimana didalam file .py berisikan script untuk membuka file .shp sama seperti script di atas. Penutup: Kesimpulan: Dengan script diatas kita dapat mengetahui jumlah record yang ada pada file shp yang kita buka Saran: Sebaiknya langsung dipraktekkan dan dipahami ,atau bisa dibandingkan dengan membuka file shp tersebut menggunakan QGIS.
